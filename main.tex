%! Author = robin
%! Date = 31.03.2020

% Preamble
\documentclass{article}

% Packages
\usepackage{amsmath}
\usepackage{subfigure}
\usepackage{graphicx}
\usepackage {tikz}
\usetikzlibrary {positioning}


\definecolor {processblue}{cmyk}{0.96,0,0,0}

% Document
\begin {document}

    \title{Inzidenz- und Adjazenzmatrizen von Graphen}
    \author{Robin Berberat}
    \maketitle

    \begin{figure}[h]
        \centering
        \begin {tikzpicture}[-latex ,auto ,node distance =4 cm and 5cm ,on grid ,
        semithick ,
        state/.style ={ circle ,top color =white , bottom color =white ,
        draw,black , text=black , minimum width =1 cm}]
            \node[state] (A) {$v_1$};
            \node[state] (B) [right = of A] {$v_2$};
            \node[state] (C) [below = of A] {$v_3$};
            \node[state] (D) [right = of C] {$v_4$};
            \path (A) edge [bend right = 15]  node[below =0.15 cm] {$e_1$} (B);
            \path (A) edge node[below =0.15 cm] {$e_2$} (D);
            \path (B) edge [bend right = 15]  node[above =0.15 cm] {$e_3$} (A);
            \path (C) edge node[left =0.15 cm] {$e_4$} (A);
            \path (D) edge node[right =0.15 cm] {$e_5$} (B);
        \end{tikzpicture}
        \caption{Beispiel eines Graphen}
    \end{figure}

    \section{Inzidenzmatrizen}
    Eine Inzidenzmatrix eines Graphen, ist eine Matrix, welche die Beziehungen zwischen der Knoten und Kanten des Graphen aufzeichent.
    Bei n-Knoten und m-Kanten ergibt das eine \(n\times m\)-Matrix. Aus den Werten der Zeilen und Spalten kann man ablesen, welche Knoten \"uber welche Kante miteinander verbunden sind.
    \begin{figure}[h]
        \centering
        \bordermatrix{
            &e_1&e_2&e_3 &e_4&e_5  \cr
            v_1& 1 & 1 & -1 & -1 & 0 \cr
            v_2& -1 & 0 & 1 & 0 & -1 \cr
            v_3& 0 & 0 & 0 & 1 & 0  \cr
            v_4& 0 & -1 & 0 & 0 & 1}
        \caption{Inzidenzmatrix des Beispielgraphen}
    \end{figure}

    \section{Adjazenzmatrizen}
    Eine Adjazenzmatrix eines Graphen, ist eine Matrix, welche die Verbindungen zwischen der Knoten des Graphen aufzeichent.
    Bei n-Knoten ergibt das eine \(n\times n\)-Matrix. Aus den Werten der Zeilen und Spalten kann man ablesen, welche Knoten miteinander verbunden sind.

    \begin{figure}[h]
        \centering
        \bordermatrix{
            &v_1&v_2&v_3 &v_4  \cr
            v_1& 0 & 1 & 0 & 1 \cr
            v_2& 1 & 0 & 0 & 0 \cr
            v_3& 1 & 0 & 0 & 0  \cr
            v_4& 0 & 1 & 0 & 0 }
        \caption{Adjazenzmatrix des Beispielgraphen}
    \end{figure}


\end{document}